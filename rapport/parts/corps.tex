\section{Les différentes parties}
Afin d'expliciter la logique de mon programme, je vous propose de suivre l'exécution de la méthode \textit{main}. 
Nous commençons donc par charger les produits en mémoire, et nous en profitons pour les trier (ce qui n'a, pour cette échéance ci, aucun intérêt. Cependant, j'aime à croire que je pourrais améliorer la vitesse d'exécution de l'algorithme proposé sur base de ça pour la prochaine échéance. C'est, de plus, l'occasion d'utiliser une méthode de haut niveau ainsi qu'une fonction anonyme).

\subsection{Les entrées utilisateurs}
Nous arrivons ensuite à la première partie du programme, à savoir, les informations provenant de l'utilisateur. Puisque nous cherchons à calculer la quantité de calories maximum que celui-ci doit ingérer par repas, nous devons donc savoir s'il s'agit d'un homme ou d'une femme, en plus du type de repas (déjeuner, diner ou souper), mais nous y reviendrons plus tard ; concentrons nous dans un premier temps sur le sexe de l'utilisateur. Pour ce faire, j'ai donc décidé de déclarer un \textit{Trait}, \textit{Human}, et de faire ainsi jouer l'héritage de Scala en déclarant deux classes implémentant ces traits (à savoir \textit{Man} et \textit{Woman}). La même logique fut utilisé pour le type de repas. 

Sur bases de ces informations, nous pouvons obtenir la quantité de calories maximum par type de repas. Pour ce faire, je me suis basé sur une règle explicitant qu'il faille manger 1/6 des calories de la journée (2500 pour un homme, 2000 pour une femme) au matin, 1/4 lors du diner et 1/4 au souper (le \~ tier restant étant ingéré via les gourmandises et les boissons).  

Maintenant que nous avons représenté le sexe et le type de repas, nous devons demander à l'utilisateur de nous indiquer ces informations. C'est pourquoi, en suivant la même nomenclature que le singleton déjà déclaré dans la \textit{code base} (\textit{GastroExtractor}), j'ai implémenté l'objet \textit{GastroUser}. Les méthodes récursives \textit{get\_sex} et \textit{get\_meal}, sur base de \textit{pattern matching}, nous permettent ainsi de récupérer ces informations.
 
\subsection{L'algorithme de composition}
Une fois ces divers informations récupérées, nous pouvons maintenant nous atteler à la composition du menu. C'est pourquoi nous instancions la classe \textit{MenuComposer}. Dedans, nous retrouvons 4 méthodes privées en plus de la principale, à savoir \textit{compose}. Ici, nous indiquons à l'utilisateur la valeur maximum qu'il devrait respecter, puis nous lui indiquons un mélange de 3 ingrédients choisis via la méthode \textit{not\_so\_random\_products}. Cette dernière est très simple, elle ne fait que de choisir 3 produits aléatoirement, mais ne dépassant pas - individuellement - le maximum autorisé pour le repas. Si les 3 produits pris ensembles excédent le nombre maximum de calories autorisées, celle-ci se \textit{ré-appelle} elle même, autrement, elle aura pour valeur de retour ces trois produits choisis.

Nous finissons par présenter les 3 produits à l'utilisateur et lui demandons s'il veut obtenir un autre résultat. A noter que je me suis permis de remplacer la boucle \textit{while} par une fonction récursive afin d'emprunter moins à la programmation déclarative.