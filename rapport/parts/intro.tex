\vfill
\begin{figure}[H]
    \centering
    \includegraphics[width=1\textwidth]{parts/pics/2.png}
    \caption{Aperçu du programme.}
    \label{fig:my_label}
\end{figure}

\newpage
\section{Introduction}
Etant donné que je ne suis pas un expert en alimentation, et que l'objectif transversale du projet était de montrer la compréhension des concepts du langage Scala (à savoir un mélange de fonctionnel et d'orienté objet), mon script pourra vous paraître simpliste. 

En effet, il ne se base que sur le nombre de calories présent dans le repas afin de déterminer si celui-ci convient ou pas. Cependant, j'ai préféré garder un programme simple afin de mettre en oeuvre les deux paradigmes de Scala et ainsi mieux appréhender l'apprentissage de ce langage. 

Le code étant lisible et commenté, je ne m'attarderai pas sur l'aspect technique du programme dans ce rapport, mais seulement la partie logique.

\begin{figure}[H]
    \centering
    \includegraphics[width=1\textwidth]{parts/pics/1.png}
    \caption{La compilation n'émet aucun \textit{Warning}.}
    \label{fig:my_label}
\end{figure}